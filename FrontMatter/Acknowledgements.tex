% ------------------------------------------------------------------------ %
% !TEX encoding = UTF-8 Unicode
% !TEX TS-program = pdflatex
% !TEX root = ../Tesi.tex
% !TEX spellcheck = it-IT
% ------------------------------------------------------------------------ %
%
% ------------------------------------------------------------------------ %
% 	RINGRAZIAMENTI
% ------------------------------------------------------------------------ %
%
\cleardoublepage
%
\phantomsection
%
\pdfbookmark{Acknowledgements}{Acknowledgements}
%
\chapter*{Acknowledgements}
%
Questa tesi rappresenta il punto di arrivo del lungo viaggio che mi ha portata dall'essere una ragazzina impaurita di 19 anni, arrivata a Milano senza un'idea definita di dove stesse andando, alla donna consapevole e (perché no?) fiera che sono oggi. Se qualcuno avesse chiesto a quella ragazzina dove questo viaggio l'avrebbe condotta, probabilmente lei non avrebbe saputo rispondere. Se qualcuno le avesse chiesto dove \emph{sperava} che questo viaggio la conducesse, probabilmente la risposta sarebbe stata \emph{qui, dove sono oggi}. 

Grazie alla mia esperienza al Politecnico di Milano, sono riuscita a scoprire e applicare una passione che non sapevo neanche di avere, quindi, nonostante non sia una persona, il primo che voglio ringraziare per questo traguardo è sicuramente il \emph{Poli}, che, con le sue difficoltà e le sue opportunità, mi ha insegnato a non demordere se la ricompensa attesa è la propria soddisfazione. A questo riguardo, ringrazio mio fratello Emanuele, senza il cui intervento non avrei neanche sostenuto i test d'ingresso, e i miei genitori, che mi hanno sempre sostenuta e motivata non solo per tutta la durata del corso di studi, ma anche di tutta la mia vita.

Ringrazio tutti i miei compagni di corso, che mi hanno accompagnata nel viaggio e hanno condiviso con me gioie e dolori, in particolare Silvia, \emph{Fork}, la quale vicina e lontana è sempre stata presente. Ringrazio Sara, che non si è fermata alle apparenze, mi ha aiutata ad aprirmi e non mi ha mai mollata da allora. Le mie amiche Giulia, Martina, Chiara ed Eleonora e la mia coinquilina Simona, sulle quali ho sempre potuto contare. Ringrazio il mio ragazzo, Giovanni: nonostante il poco tempo e la distanza, la tua presenza è stata enorme.

Infine, un ringraziamento va al mio relatore, William Fornaciari, che mi ha permesso di conoscere i ragazzi dell'\emph{HeapLab}, in particolare Giuseppe, che mi ha seguita nel corso di questa tesi come un amico prima che come un correlatore: grazie per avermi accompagnata verso il traguardo.
\medskip



\bigskip
 
\noindent\textit{\myLocation, \myTime}
\hfill M.~P.
%
% ------------------------------------------------------------------------ %